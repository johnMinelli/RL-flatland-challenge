\section{GNN}
Given the use of a graph observation, we needed a mechanism to normalize it in order to then feed it to the reinforcement learning algorithm. GNN are typically used to create an embedded representation of node features, which is then used for another task, like classification or regression. \\ \\

\noindent
Specifically, we wanted to fully take advantage of the connections between the nodes, so that every node's representation, after being processed by the GNN, would in fact include data about the neighborhood. Thus, we employed \textbf{Graph Convolutional Networks}, as described in \cite{gcn}: GCN perform a convolution on graph features, by using an update rule 
$$H^{(l + 1)} = \sigma(D^{\frac{-1}{2}} A' D^{\frac{-1}{2}} H^{l} W )$$ that, at each layer, effectively combines together features of adjacent nodes.\\ $H^0$ is the input feature matrix that contains features for each node, $W$ is a layer-specific weight matrix, while the \textbf{adjacency matrix} $A$ is actually summed with the idendity matrix in order to obtain $A'$, which considers also self-connections in the graph, otherwise each node in the GCN's output would never include information about themselves. \\
Furthermore, the \textbf{degree matrix} $D$ is included for \textit{symmetric normalization}, since the matrix multiplication risks scaling up the features.\\
Everything is then passed to an non-linear activation function $\sigma$, and repeated for a number of convolution layers which is a specific parameter of the model.\\ \\

\begin{figure}[H] 
\includegraphics[height=80mm, width=140mm, scale=0.5]{figures/gcn.png}
\centering
\caption{GCN}
\label{fig:s5} 
\end{figure}
\noindent
In \cite{a2c}, in order to summarize the DAG, the authors chose to include, as features and for each node, the number of successors and predecessors nodes, as well as some task scheduling-specific variables. Thus we decided that each node should have
\begin{itemize}
\item the minimum of all \textbf{distances} from neighbors in the graph observation, since it expresses better than number of predecessors and successors the ''position'' of a node in the DAG, especially w.r.t. to the target.
\item a one-hot encoding of its type: \textit{starting} or target node, \textit{conflict}, \textit{deadlock}, \textit{starvation} node, a mix of those ( a node could be a conflict but also the starting one) or \textit{other}, in case it doesn't belong to any other category.

\end{itemize}
\noindent
After the input node features have been processed by the GCN, agent-specific and environmental features, like velocity, number of malfunctions, malfunction rate are added to the vector representation that is then passed to the DQN or the A2C.
